\section{Execution}
\label{sec:execution}

In this section we take an introspective look at the execution of the queries.
We do this to check if the system is able to perform an efficient computation, or if we can further optimize the queries by changing some code.
Let's start by looking at the physical plan, obtained by calling \texttt{DataSet.explain()}.

\subsection{Physical plan}

The first three operations just take care of the loading and preprocessing of the data — in this case we are using the ECDC format.

\begin{lstlisting}[style=sparkplan]
(1) Scan csv 
Output [3]: [dateRep#0, cases#4, countriesAndTerritories#6]
Batched: false
Location: InMemoryFileIndex [file:data\data.csv]
PushedFilters: [IsNotNull(dateRep)]
ReadSchema: struct<dateRep:date,cases:int,countriesAndTerritories:string>

(2) Filter [codegen id : 1]
Input [3]: [dateRep#0, cases#4, countriesAndTerritories#6]
Condition : isnotnull(dateRep#0)

(3) Project [codegen id : 1]
Input [3]: [dateRep#0, cases#4, countriesAndTerritories#6]
Output [3]: [dateRep#0 AS date#24, cases#4, countriesAndTerritories#6 AS country#25]
\end{lstlisting}

\noindent
Then operations~(4) to~(9) create the sliding window using \texttt{Expand}, and produce the average for each window using HashAggregate. We also note that to compute the aggregation function an \texttt{Exchange} (i.e.\ a shuffle operation) has to be performed, to partition the dataset by $\langle country, window\rangle$.

\begin{lstlisting}[style=sparkplan]
(4) Expand [codegen id : 1]
Input [3]: [date#24, cases#4, country#25]
Arguments: [...], [window#35, date#24, cases#4, country#25]

(5) Filter [codegen id : 1]
Input [4]: [window#35, date#24, cases#4, country#25]
Condition : ((isnotnull(window#35) AND (cast(date#24 as timestamp) >= window#35.start)) AND (cast(date#24 as timestamp) < window#35.end))

(6) Project [codegen id : 1]
Input [4]: [window#35, date#24, cases#4, country#25]
Output [3]: [window#35, cases#4, country#25]

(7) HashAggregate [codegen id : 1]
Input [3]: [window#35, cases#4, country#25]
Keys [2]: [country#25, window#35]
Functions [1]: [partial_sum(cast(cases#4 as bigint))]
Aggregate Attributes [1]: [sum#65L]
Results [3]: [country#25, window#35, sum#66L]

(8) Exchange
Input [3]: [country#25, window#35, sum#66L]
Arguments: hashpartitioning(country#25, window#35, 200), ENSURE_REQUIREMENTS, [id=#49]

(9) HashAggregate [codegen id : 2]
Input [3]: [country#25, window#35, sum#66L]
Keys [2]: [country#25, window#35]
Functions [1]: [sum(cast(cases#4 as bigint))]
Aggregate Attributes [1]: [sum(cast(cases#4 as bigint))#33L]
Results [4]: [cast(window#35.start as date) AS start_date#39, cast(window#35.end as date) AS end_date#40, country#25, (cast(sum(cast(cases#4 as bigint))#33L as double) / 7.0) AS avg#34]
\end{lstlisting}

\noindent
The next set of operations compute the percentage increase.
Now the data must be partitioned by country, hence the \texttt{Exchange} operation.
However, this second \texttt{Exchange} is undesirable as moving data between nodes is very costly. In our opinion, it could be avoided by partitioning only by $country$ at~(8), since in theory the \texttt{HashAggregate} at~(9) requires data to be partitioned by at least one attribute in the key.

We tried to partition the dataset by $country$ before grouping it by $\langle country, window\rangle$, which introduces an \texttt{Exchange} before~(4), but doesn't remove any of the subsequent ones. We suspect that the partitioning information is not propagated correctly, or Spark is unable to perform such a fine-grained optimization.

\begin{lstlisting}[style=sparkplan]
(10) Exchange
Input [4]: [start_date#39, end_date#40, country#25, avg#34]
Arguments: hashpartitioning(country#25, 200), ENSURE_REQUIREMENTS, [id=#53]

(11) Sort [codegen id : 3]
Input [4]: [start_date#39, end_date#40, country#25, avg#34]
Arguments: [country#25 ASC NULLS FIRST, end_date#40 ASC NULLS FIRST], false, 0

(12) Window
Input [4]: [start_date#39, end_date#40, country#25, avg#34]
Arguments: [lag(avg#34, -1, null) windowspecdefinition(country#25, end_date#40 ASC NULLS FIRST, specifiedwindowframe(RowFrame, -1, -1)) AS avg_lag#47], [country#25], [end_date#40 ASC NULLS FIRST]

(13) Project [codegen id : 4]
Input [5]: [start_date#39, end_date#40, country#25, avg#34, avg_lag#47]
Output [5]: [start_date#39, end_date#40, country#25, avg#34, (((avg#34 - avg_lag#47) / avg_lag#47) * 100.0) AS avg_increase#53]

(14) InMemoryRelation
Arguments: [start_date#39, end_date#40, country#25, avg#34, avg_increase#53], CachedRDDBuilder(...,StorageLevel(disk, memory, deserialized, 1 replicas)]
\end{lstlisting}

\pagebreak\noindent
Now we intentionally leave out the part related to coalescing the partitions and saving to file, and we turn to the computation of the rank.
As made clear by~(15) the data being used is the one cached at~(14).

The records are re-partitioned according to the period (either $start~date$ or $end~date$ can be used as key), they are sorted by $avg~increase$, the rank is computed and finally cut to 10.
Note that the \texttt{Exchange} operation at~(16) is unavoidable, since the previous partitioning was by $country$.

\begin{lstlisting}[style=sparkplan]
(15) InMemoryTableScan
Output [5]: [start_date#39, end_date#40, country#25, avg#34, avg_increase#53]
Arguments: [start_date#39, end_date#40, country#25, avg#34, avg_increase#53]

(16) Exchange
Input [5]: [start_date#39, end_date#40, country#25, avg#34, avg_increase#53]
Arguments: hashpartitioning(end_date#40, 200), ENSURE_REQUIREMENTS, [id=#83]

(17) Sort [codegen id : 1]
Input [5]: [start_date#39, end_date#40, country#25, avg#34, avg_increase#53]
Arguments: [end_date#40 ASC NULLS FIRST, avg_increase#53 DESC NULLS LAST], false, 0

(18) Window
Input [5]: [start_date#39, end_date#40, country#25, avg#34, avg_increase#53]
Arguments: [dense_rank(avg_increase#53) windowspecdefinition(end_date#40, avg_increase#53 DESC NULLS LAST, specifiedwindowframe(RowFrame, unboundedpreceding$(), currentrow$())) AS rank#202], [end_date#40], [avg_increase#53 DESC NULLS LAST]

(19) Filter [codegen id : 2]
Input [6]: [start_date#39, end_date#40, country#25, avg#34, avg_increase#53, rank#202]
Condition : (isnotnull(rank#202) AND (rank#202 <= 10))
\end{lstlisting}